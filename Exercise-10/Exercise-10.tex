\documentclass[11pt]{article}
\usepackage[ngerman]{babel}
\usepackage[left=2cm,right=2cm]{geometry}
\usepackage{amsmath}
\usepackage{amsthm}
\usepackage{thmtools}
\usepackage{amsfonts}
\usepackage{amssymb}
\usepackage[]{algorithm2e}
\newcommand{\numpy}{{\tt numpy}}    % tt font for numpy

\topmargin -.5in
\textheight 9in
\oddsidemargin -.25in
\evensidemargin -.25in
\textwidth 7in

\begin{document}
	
	% ========== Edit your name here
	\title{\textbf{Cryptography Exercise Sheet 10}}
	\author{Prof. Dr. Thomas Wilke, Dr.-Ing. Kim-Manuel Klein}
	\maketitle
	\medskip
	
	% ========== Begin answering questions here
	\begin{enumerate}
		
		\item
		\textbf{Problem:} \textit{Computationally unbounded adversaries (Exercise 10.1 in BS)}
		
		Show that an anonymous key exchange
		protocol $P$ (as in Definition 10.1) cannot be secure against a computationally unbounded adversary.
		This explains why all protocols in this chapter must rely on computational assumptions.
		
		
		\item
		\textbf{Problem:} \textit{Random self-reduction for CDH (I) (Exercise 10.4 in BS)}
		
		Consider a specific cyclic group $\mathbb{G}$ of prime order
		$q$ generated by $g \in \mathbb{G}$. For $u = g^\alpha \in \mathbb{G}$ and  $v=g^{\beta} \in \mathbb{G}$, define $[u, v] = g^{\alpha\beta}$, which is the solution instance $(u, v)$ of the CDH problem. Consider the randomized mapping from $\mathbb{G}^2$ to $\mathbb{G}^2$ that sends
		$(u, v)$ to $(\tilde{u}, v)$, where
		\begin{center}
			 $p \xleftarrow{R} \mathbb{Z}_q$, $\tilde{u} \xleftarrow{} g^p u$.
		\end{center}
		Show that 
		
		\begin{enumerate}
			\item $\tilde{u}$ is uniformly distributed over $\mathbb{G}$;
			\item $[\tilde{u}, v] = [u,v] \cdot v^p$.
		\end{enumerate}
		
		
		\item
		\textbf{Problem:} \textit{Problems equivalent to CDH (Exercise 10.15/10.16 in BS)}
		
		Consider a specific cyclic group $\mathbb{G}$ of prime order $q$ generated by $g \in \mathbb{G}$. Show that the following problems are deterministic poly-time equivalent:
		\begin{enumerate}
			\item Given $g^{\alpha}$ and $g^{\beta}$, compute $g^{\alpha\beta}$ (this is just the Computational Diffie-Hellman problem).
			\item Given $g^{\alpha}$, compute $g^{(\alpha^2)}$.
			\item Given $g^{\alpha}$ with $\alpha \ne 0$, compute $g^{1/\alpha}$.
			\item Given $g^{\alpha}$ and $g^{\beta}$ with $\beta \ne 0$, compute $g^{\alpha/\beta}$.
			
		\end{enumerate}
	
		Note that all problem instances are defined with respect to the same group $\mathbb{G}$ and generator $g \in \mathbb{G}$.
		
		\item
		\textbf{Problem:} \textit{A proper trapdoor permutation scheme based on RSA (Exercise 10.24/10.25 in BS)}
		
		As discussed in Section 10.3, our RSA-based trapdoor permutation scheme does not quite satisfy our definitions,
		simply because the domain on which it acts varies with the public key. This exercise shows one way
		to patch things up. Let $\ell$ and $e$ be parameters used for RSA key generation, and let $G$ be the key
		generation algorithm, which outputs a pair $(pk , sk )$. Recall that $pk = (n, e)$, where $n$ is an RSA
		modulus, which is the product of two $\ell$-bit primes, and $e$ is the encryption exponent. The secret
		key is $sk = (n, d)$, where $d$ is the decryption exponent corresponding to the encryption exponent $e$.
		Choose a parameter $L$ that is a substantially larger than 2$\ell$, so that $n/2^L$ is negligible. Let $\mathcal{X}$ be
		the set of integers in the range $[0, 2^L )$. We shall present a trapdoor permutation scheme $(G, F^{\ast} , I^{\ast} )$, defined over $\mathcal{X}$. The function $F^{\ast}$ takes two inputs: a public key $pk$ as above and an integer $x \in \mathcal{X}$ ,
		and outputs an integer $y \in \mathcal{X}$, computed as follows. Divide $x$ by $n$ to obtain the integer quotient
		$Q$ and remainder $R$, so that $x = nQ + R$ and $0 \leq R < n$. If $Q > 2^L / n - 1$, then set $S := R$;
		otherwise, set $S := R^e$ mod $n$. Finally, set $y := nQ + S$.
		
		\begin{enumerate}
			\item Show that $F^{\ast} (pk, \cdot)$ is a permutation on $\mathcal{X}$, and give an efficient inversion function $I^{\ast}$ that
			satisfies $I^{\ast} (sk , F^{\ast} (pk , x)) = x$ for all $x \in \mathcal{X}$.
			\item Show under the RSA assumption, $(G, F^{\ast} , I^{\ast})$ is one-way.
		\end{enumerate}
		
		
		
	\end{enumerate}
	
\end{document}
\grid
\grid