\documentclass[11pt]{article}
\usepackage[ngerman]{babel}
\usepackage[left=2cm,right=2cm]{geometry}
\usepackage{amsmath}
\usepackage{amsthm}
\usepackage{thmtools}
\usepackage{amsfonts}
\usepackage{amssymb}
\usepackage[]{algorithm2e}
\newcommand{\numpy}{{\tt numpy}}    % tt font for numpy

\topmargin -.5in
\textheight 9in
\oddsidemargin -.25in
\evensidemargin -.25in
\textwidth 7in

\begin{document}
	
	% ========== Edit your name here
	\title{\textbf{Cryptography Exercise Sheet 5}}
	\author{Prof. Dr. Thomas Wilke, Dr.-Ing. Kim-Manuel Klein}
	\maketitle
	\medskip
	
	% ========== Begin answering questions here
	\begin{enumerate}
		
		\item
		\textbf{Problem:} \textit{Double encryption (Exercise 5.1 in BS)}
		
		Let $\mathcal{E} = (E,D)$ be a cipher. Consider the cipher $\mathcal{E}_2=(E_2, D_2)$, where $E_2(k,m) = E(k, E(k,m))$. One would expect that if encrypting a message once with $E$ is secure then encrypting it twice as in $E_2$ should be no less secure. However, that is not always true.
		\begin{enumerate}
		
 		\item Show that there is a semantically secure cipher $ \mathcal{E}$ such that $\mathcal{E}_2$ is not semantically secure.
		\item Prove that for every CPA secure ciphers $\mathcal{E}$, the cipher $\mathcal{E}_2$ is also CPA secure. That is, show
		that for every CPA adversary $ \mathcal{A}$ attacking $\mathcal{E}_2$ there is a CPA adversary $ \mathcal{B}$ attacking $ \mathcal{E}$ with
		about the same advantage and running time.
		
     	\end{enumerate}
	
		\item
        \textbf{Problem:} \textit{An alternate definition of CPA security (Exercise 5.3 in BS)}
        
        This exercise develops an alternative characterization of CPA security for a cipher $\mathcal{E} = (E,D)$, defined over  $(\mathcal{K}, \mathcal{M}, \mathcal{C})$. As usual, we need to
        define an attack game between an adversary $\mathcal{A}$ and a challenger. Initially, the challenger generates
      	\begin{align}
      	b \xleftarrow{\text{R}} \{0,1\}, k \xleftarrow{\text{R}} \mathcal{K}. 
      	\end{align}
      
      
        Then $\mathcal{A}$ makes a series of queries to the challenger. There are two types of queries:
        \begin{description}
        	\item[Encryption:] In an \textit{encryption query}, $\mathcal{A}$ submits a message $m  \in \mathcal{M}$ to the challenger, who responds with a ciphertext $c \xleftarrow{\text{R}} E(k, m)$. The adversary may make any (poly-bounded) number of encryption queries.
        	\item[Test:] In a \textit{test query}, $\mathcal{A}$ submits a pair of messages $m_0, m_1 \in \mathcal{M}$ to the challenger, who responds with a ciphertext $c \xleftarrow{\text{R}} E(k, m_b)$. The adversary is allowed to make only a \textit{single} test query (with any number of encryption queries before and after the test query).
        \end{description}

        At the end of the game, $\mathcal{A}$ outputs a bit $\hat{b} \in \{0,1\}$.
        
        
        As usual, we define $\mathcal{A}$’s advantage in the above attack game to be $|Pr[\hat{b}= b]-1/2|$. We say that
        $\mathcal{E}$ is Alt-CPA secure if this advantage is negligible for all efficient adversaries.
        
        Show that $\mathcal{E}$ is CPA secure if and only if $\mathcal{E}$ is Alt-CPA secure.
		
		\item
		\textbf{Problem:} \textit{Ciphertext expansion vs. security (Exercise 5.10 in BS)}
			
		Let $\mathcal{E} = (E,D)$ be an encryption scheme where	messages and ciphertexts are bit strings.
		\begin{enumerate}
	
		 \item Suppose that for all keys and all messages $m$, the encryption of $m$ is the exact same length as $m$. Show that $(E, D)$ cannot be semantically secure under a chosen plaintext attack.
		\item Suppose that for all keys and all messages $m$, the encryption of m is exactly $\ell$ bits longer
		than the length of $m$. Show an attacker that can win the CPA security game using $\approx 2^{\ell/2}$
		queries and advantage $\approx$ 1/2. You may assume the message space contains many more than
		$2^{\ell/2}$ messages.
		\end{enumerate}
	\end{enumerate}
	
\end{document}
\grid
\grid