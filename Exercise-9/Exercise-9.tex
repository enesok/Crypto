\documentclass[11pt]{article}
\usepackage[ngerman]{babel}
\usepackage[left=2cm,right=2cm]{geometry}
\usepackage{amsmath}
\usepackage{amsthm}
\usepackage{thmtools}
\usepackage{amsfonts}
\usepackage{amssymb}
\usepackage[]{algorithm2e}
\newcommand{\numpy}{{\tt numpy}}    % tt font for numpy

\topmargin -.5in
\textheight 9in
\oddsidemargin -.25in
\evensidemargin -.25in
\textwidth 7in

\begin{document}
	
	% ========== Edit your name here
	\title{\textbf{Cryptography Exercise Sheet 9}}
	\author{Prof. Dr. Thomas Wilke, Dr.-Ing. Kim-Manuel Klein}
	\maketitle
	\medskip
	
	% ========== Begin answering questions here
	\begin{enumerate}
		
		\item
		\textbf{Problem:} \textit{AE-security: simple examples (Exercise 9.1 in BS)}
		
		Let $(E, D)$ be an AE-secure cipher. Consider the following derived ciphers:
		
		\begin{enumerate}
			\item $E_1(k,m) := (E(k,m), E(k,m))$;                                  $D_1(k, (c_1, c_2)) := \begin{cases} D(k,c_1) & \text{if $D(k,c_1) = D(k,c_2)$ }\\ {\bf reject}& \text{otherwise}\end{cases}$
			
			\item $E_2(k,m) := \{c \xleftarrow{}E(k,m),  output(c,c) \}$; $D_2(k, (c_1, c_2)) := \begin{cases} D(k,c_1) & \text{if $c_1 = c_2$ }\\ {\bf reject}& \text{otherwise}\end{cases}$
		\end{enumerate}
	
	Show that part (b) is AE-secure, but part (a) is not.
	\end{enumerate}
	
\end{document}
\grid
\grid