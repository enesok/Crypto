\documentclass[11pt]{article}
\usepackage[ngerman]{babel}
\usepackage[left=2cm,right=2cm]{geometry}
\usepackage{amsmath}
\usepackage{amsthm}
\usepackage{thmtools}
\usepackage{amsfonts}
\usepackage{amssymb}
\usepackage[]{algorithm2e}
\newcommand{\numpy}{{\tt numpy}}    % tt font for numpy

\topmargin -.5in
\textheight 9in
\oddsidemargin -.25in
\evensidemargin -.25in
\textwidth 7in

\begin{document}
	
	% ========== Edit your name here
	\title{\textbf{Solutions to Shannon Ciphers and Perfect Security}}
	\author{Prof. Dr. Thomas Wilke, Dr.-Ing. Kim-Manuel Klein}
	\maketitle
	\medskip
	
	% ========== Begin answering questions here
	\begin{enumerate}
		
		\item
	\textbf{Problem:} \textit{Multiplicative one-time pad}
		
	Correctness: $D(k,E(k,m))= D(k,k \cdot m \mod p) = k^{-1 (p)} \cdot (k \cdot m \mod p) \mod p = k^{-1 (p)} \cdot k \cdot m \mod p = m$
	
	
	Perfect security: $\Pr[E(k,m_0) = c] = \Pr[k \cdot m_0 = c] = \frac{1}{p-1} = \Pr[k \cdot m_1 = c] = \Pr[E(k,m_1) = c]$.\\
	
	
	%Possibilistic security can easily be verified by setting $k = c \cdot m^{-1 (p)}$.
	For every fixed $m \in \mathcal{M}$, consider the function $e(k) = E(k,m)$.
	The fundamental principle of the one-time pad is that we have for each cipher text $c \in \mathcal{C}$ a key $k \in \mathcal{K}$ such that $e(k) = c$ and vice versa. Hence $e$ is a bijection. This is also the case for the above multiplicative one-time pad as $k$ can be chosen by $k = c \cdot m^{-1 (p)}$ for every $c \in \mathcal{C}$ and hence
	\begin{align*}
		e(k) \equiv km \equiv c \cdot m^{-1 (p)} m \equiv c \mod p
	\end{align*}
		
	
		\item
       \textbf{Problem:} \textit{A good substitution cipher}
        
         Let $m_0,m_1 \in \mathcal{M}$ and $c[0] , \ldots , c[L-1] \in \mathcal{C}$
        \begin{align*}
        	\Pr[E(k,m_0) = c[0] , \ldots , c[L-1]] \\= \Pr[k[0](m_0[0]), \ldots , k[L-1](m_0[L-1]) = c[0] , \ldots , c[L-1]] \\= \Pr[k[0](m_0[0]) = c[0]] \cdot \ldots \cdot, \Pr[k[L-1](m_0[L-1]) = c[L-1]]
        \end{align*}
        The probability that a random permutation maps an element $x \in \Sigma$ to $c[i]$ is exactly $\frac{1}{|\Sigma|}$, i.e. $\Pr[k[i](m_0[i]) = c[i]] = \frac{1}{|\Sigma|} = \Pr[k[i](m_1[L-1]) = c[i]]$.
        Hence we obtain
        \begin{align*}
        	\Pr[k[0](m_0[0]) = c[0]] \cdot \ldots \cdot \Pr[k[L-1](m_0[L-1]) = c[L-1]] \\
        	= \Pr[k[0](m_1[0]) = c[0]] \cdots \Pr[k[L-1](m_1[L-1]) = c[L-1]] \\
        	= \Pr[E(k,m_1) = c[0] , \ldots , c[L-1]]
        \end{align*}
    
		\item
		\textbf{Problem:}\textit{ A broken one-time pad}
		
	   	\begin{itemize}
			\item Let $k \in \mathcal{K}$ be uniformly chosen, then $\Pr[E(0^L,k) = 0^{L-1}1] = 0$ as it is not possible to produce a string with an odd number of $1'$s from $0^L$. However, the probability for $m_1 = 0^{L-1}1$ is $\Pr[E(k,0^{L-1}1) = 0^{L-1}1] = \frac{1}{2^{L-1}}$ as $0^{L-1}1 \oplus 0^{L} = 0^{L-1}1$, where $0^L$ is a key with an even number of $1$'s. Hence this variant of the one-time pad is not perfectly secure.
			\item Consider the property $\Phi(c) = \begin{cases} 1 & \text{if $c$ contains an odd number of $1$'s}\\ 0 & \text{otherwise}\end{cases}$
			
			By the same argumentation as above, we obtain that $\Pr[\Phi(E(k,0^L))] = 0$ while $\Pr[\Phi(E(k,0^{L-1}1))] = 1$. Hence this variant of the one-time pad is not semantically secure.
			\item The probability distribution of $E(k,0^L)$ is different to the probability distribution of $E(k,0^{L-1}1)$ as for example, by the argumentation in the first part, $\Pr[E(0^L,k) = 0^{L-1}1] = 0$ while $\Pr[E(k,0^{L-1}1) = 0^{L-1}1] = \frac{1}{2^{L-1}}$. Hence this variant of the one-time pad is not ciphertext independent.
		\end{itemize}
	\end{enumerate}
	
\end{document}
\grid
\grid